\chapter*{Abstract}%
The Simultaneous Localization and Mapping (SLAM) problem for mobile robots aims at consistently building a map of an unknown environment while simultaneously determining its position within this map. From a control-theoretic viewpoint, it is somehow analogous to simultaneously estimating the states and output map of the system. In the robotics community, SLAM is arguably considered a solved problem on a theoretical and conceptual level, but still it requires considerable maturity on a practical level. The state-of-the-art SLAM algorithms require computationally powerful processors, expensive sensors with dense feature extraction and multiple sensors for uncertainty reduction. An approach to the SLAM problem using minimal sensing information is still lacking in both theoretical and practical aspects. 

In this context, this M.Sc.\ thesis aims to study and implement a special type of SLAM solely using the impact information from a spherical mobile robot called Sphero (developed by Orbotix). Such impact data is available from the robot's onboard \acf{IMU}, however the impact angle and odometry information are subject to significant drift. Thus, the SLAM problem will be restricted to a rectilinear environment in order to allow for calibration and correction of such accumulated IMU errors (assuming that impacts with walls are sufficiently frequent). 

The impact-based SLAM is an observable estimation problem with downside of a poor robot pose distribution. An accurate representation of the pose distribution is a particle set, with the resulting estimation technique as particle filter. Suitable map representations for the impact-based SLAM problem are formulated and studied, and the most efficient one is implemented. A probabilistic formulation is laid out for the SLAM problem using robot motion model and map representation, and associated challenges are studied for developing an efficient algorithm.

The resulting SLAM algorithm uses a Rao-Blackwellized particle filter which is computationally efficient and robust to data association errors. The issue of inconsistency is discussed for the developed SLAM algorithm and suitable modifications are proposed over the developed algorithm for ensuring consistency. SLAM is extended to multiple robots as a map-merging problem since multiple robots can build a perceptually rich map with a lower exploration time.