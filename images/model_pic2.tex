\documentclass{article}
\usepackage[latin1]{inputenc}
\usepackage{tikz}
\usetikzlibrary{shapes.geometric, arrows, calc}

% Defining environment
\usepackage{verbatim}
\usepackage[active,tightpage]{preview} % Cropping image to full size
\PreviewEnvironment{tikzpicture} % Picture as pdf
\setlength\PreviewBorder{50pt}%

\tikzstyle{block} = [rectangle, draw=black, fill=orange!30,
    text width=8em, text centered, rounded corners, minimum width=5em, minimum height=1cm]
\tikzstyle{line} = [draw, very thick, color=black!70, -latex']
\tikzstyle{io} = [trapezium, trapezium left angle=70, trapezium right angle=110, text width=4em,
minimum width=2em, minimum height=1cm, text centered, draw=black, fill=blue!30]
\tikzstyle{map} = [rectangle, draw=black, fill=red!60,text width=6em, text centered, rounded corners, minimum width=3cm, minimum height=2cm]

\begin{document}
\begin{tikzpicture}[scale=2, node distance = 2.5cm, auto]
	\node[io](input){Odometry};
	\node[block, right of=input,xshift=2cm](SLAM){SLAM algorithm};
	\node[io, below of=SLAM](measurement){Measurement};
	\node[map, right of=SLAM,xshift=2cm](Output){Map and Robot trajectory};
	
	\path[line] (measurement) -- node[anchor=east]{$y'$}(SLAM);
	\path[line] (input) -- node[anchor=south]{$u'$}(SLAM) ;	
	\path[line] (SLAM) -- (Output);
	
\end{tikzpicture}
\end{document}